% Options for packages loaded elsewhere
\PassOptionsToPackage{unicode}{hyperref}
\PassOptionsToPackage{hyphens}{url}
\PassOptionsToPackage{dvipsnames,svgnames,x11names}{xcolor}
%
\documentclass[
  sigconf]{acmart}
\usepackage{amsmath}
\usepackage{lmodern}
\usepackage{iftex}
\ifPDFTeX
  \usepackage[T1]{fontenc}
  \usepackage[utf8]{inputenc}
  \usepackage{textcomp} % provide euro and other symbols
\else % if luatex or xetex
  \usepackage{unicode-math}
  \defaultfontfeatures{Scale=MatchLowercase}
  \defaultfontfeatures[\rmfamily]{Ligatures=TeX,Scale=1}
\fi
% Use upquote if available, for straight quotes in verbatim environments
\IfFileExists{upquote.sty}{\usepackage{upquote}}{}
\IfFileExists{microtype.sty}{% use microtype if available
  \usepackage[]{microtype}
  \UseMicrotypeSet[protrusion]{basicmath} % disable protrusion for tt fonts
}{}
\makeatletter
\@ifundefined{KOMAClassName}{% if non-KOMA class
  \IfFileExists{parskip.sty}{%
    \usepackage{parskip}
  }{% else
    \setlength{\parindent}{0pt}
    \setlength{\parskip}{6pt plus 2pt minus 1pt}}
}{% if KOMA class
  \KOMAoptions{parskip=half}}
\makeatother
\usepackage{xcolor}
\IfFileExists{xurl.sty}{\usepackage{xurl}}{} % add URL line breaks if available
\IfFileExists{bookmark.sty}{\usepackage{bookmark}}{\usepackage{hyperref}}
\hypersetup{
  pdftitle={Dynamics in Algorithmic Recourse},
  colorlinks=true,
  linkcolor={blue},
  filecolor={Maroon},
  citecolor={Blue},
  urlcolor={Blue},
  pdfcreator={LaTeX via pandoc}}
\urlstyle{same} % disable monospaced font for URLs
\setlength{\emergencystretch}{3em} % prevent overfull lines
\setcounter{secnumdepth}{5}
% Make \paragraph and \subparagraph free-standing
\ifx\paragraph\undefined\else
  \let\oldparagraph\paragraph
  \renewcommand{\paragraph}[1]{\oldparagraph{#1}\mbox{}}
\fi
\ifx\subparagraph\undefined\else
  \let\oldsubparagraph\subparagraph
  \renewcommand{\subparagraph}[1]{\oldsubparagraph{#1}\mbox{}}
\fi


\providecommand{\tightlist}{%
  \setlength{\itemsep}{0pt}\setlength{\parskip}{0pt}}\usepackage{longtable,booktabs,array}
\usepackage{calc} % for calculating minipage widths
% Correct order of tables after \paragraph or \subparagraph
\usepackage{etoolbox}
\makeatletter
\patchcmd\longtable{\par}{\if@noskipsec\mbox{}\fi\par}{}{}
\makeatother
% Allow footnotes in longtable head/foot
\IfFileExists{footnotehyper.sty}{\usepackage{footnotehyper}}{\usepackage{footnote}}
\makesavenoteenv{longtable}
\usepackage{graphicx}
\makeatletter
\def\maxwidth{\ifdim\Gin@nat@width>\linewidth\linewidth\else\Gin@nat@width\fi}
\def\maxheight{\ifdim\Gin@nat@height>\textheight\textheight\else\Gin@nat@height\fi}
\makeatother
% Scale images if necessary, so that they will not overflow the page
% margins by default, and it is still possible to overwrite the defaults
% using explicit options in \includegraphics[width, height, ...]{}
\setkeys{Gin}{width=\maxwidth,height=\maxheight,keepaspectratio}
% Set default figure placement to htbp
\makeatletter
\def\fps@figure{htbp}
\makeatother
\newlength{\cslhangindent}
\setlength{\cslhangindent}{1.5em}
\newlength{\csllabelwidth}
\setlength{\csllabelwidth}{3em}
\newlength{\cslentryspacingunit} % times entry-spacing
\setlength{\cslentryspacingunit}{\parskip}
\newenvironment{CSLReferences}[2] % #1 hanging-ident, #2 entry spacing
 {% don't indent paragraphs
  \setlength{\parindent}{0pt}
  % turn on hanging indent if param 1 is 1
  \ifodd #1
  \let\oldpar\par
  \def\par{\hangindent=\cslhangindent\oldpar}
  \fi
  % set entry spacing
  \setlength{\parskip}{#2\cslentryspacingunit}
 }%
 {}
\usepackage{calc}
\newcommand{\CSLBlock}[1]{#1\hfill\break}
\newcommand{\CSLLeftMargin}[1]{\parbox[t]{\csllabelwidth}{#1}}
\newcommand{\CSLRightInline}[1]{\parbox[t]{\linewidth - \csllabelwidth}{#1}\break}
\newcommand{\CSLIndent}[1]{\hspace{\cslhangindent}#1}

% \ConferenceShortName{AIES '22}
% \ConferenceName{Fifth AAAI/ACM Conference on Artificial Intelligence, Ethics, and Society}

\setcopyright{acmcopyright}
\copyrightyear{2022}
\acmYear{2022}
\acmDOI{XXXXXXX.XXXXXXX}

\acmConference[AIES '22]{Fifth AAAI/ACM Conference on Artificial Intelligence, Ethics, and Society}{Oxford, UK}

\author{Patrick Altmeyer}
\affiliation{%
  \institution{Delft University of Technology}
  \city{Delft}
  \country{The Netherlands}
}
\email{p.altmeyer@tudelft.nl}
\orcid{0000-0003-4726-8613}
\makeatletter
\makeatother
\makeatletter
\@ifpackageloaded{caption}{}{\usepackage{caption}}
\AtBeginDocument{%
\ifdefined\contentsname
  \renewcommand*\contentsname{Table of contents}
\else
  \newcommand\contentsname{Table of contents}
\fi
\ifdefined\listfigurename
  \renewcommand*\listfigurename{List of Figures}
\else
  \newcommand\listfigurename{List of Figures}
\fi
\ifdefined\listtablename
  \renewcommand*\listtablename{List of Tables}
\else
  \newcommand\listtablename{List of Tables}
\fi
\ifdefined\figurename
  \renewcommand*\figurename{Figure}
\else
  \newcommand\figurename{Figure}
\fi
\ifdefined\tablename
  \renewcommand*\tablename{Table}
\else
  \newcommand\tablename{Table}
\fi
}
\@ifpackageloaded{float}{}{\usepackage{float}}
\floatstyle{ruled}
\@ifundefined{c@chapter}{\newfloat{codelisting}{h}{lop}}{\newfloat{codelisting}{h}{lop}[chapter]}
\floatname{codelisting}{Listing}
\newcommand*\listoflistings{\listof{codelisting}{List of Listings}}
\makeatother
\makeatletter
\@ifpackageloaded{caption}{}{\usepackage{caption}}
\@ifpackageloaded{subcaption}{}{\usepackage{subcaption}}
\makeatother
\makeatletter
\@ifpackageloaded{tcolorbox}{}{\usepackage[many]{tcolorbox}}
\makeatother
\makeatletter
\@ifundefined{shadecolor}{\definecolor{shadecolor}{rgb}{.97, .97, .97}}
\makeatother
\makeatletter
\makeatother
\ifLuaTeX
  \usepackage{selnolig}  % disable illegal ligatures
\fi

\title{Dynamics in Algorithmic Recourse}
\usepackage{etoolbox}
\makeatletter
\providecommand{\subtitle}[1]{% add subtitle to \maketitle
  \apptocmd{\@title}{\par {\large #1 \par}}{}{}
}
\makeatother
\subtitle{Trustworthy Artificial Intelligence for Finance and Economics}
\author{}
\date{}

\begin{document}
\maketitle

\ifdefined\Shaded\renewenvironment{Shaded}{\begin{tcolorbox}[sharp corners, breakable, borderline west={3pt}{0pt}{shadecolor}, interior hidden, frame hidden, enhanced, boxrule=0pt]}{\end{tcolorbox}}\fi

\hypertarget{introduction}{%
\section{Introduction}\label{introduction}}

Recent advances in Artificial Intelligence (AI) have propelled its
adoption in domains outside of Computer Science including Healthcare,
Bioinformatics and Genetics. In Finance, Economics and other social
sciences, applications of AI are still relatively limited.
Decision-making in these fields has traditionally been guided by
interpretable models that facilitate explanations. Explainability is
crucial in this context, since decision-makers are typically held
accountable by the public: central banks, for example, are heavily
scrutinized for the policies they impose. It is therefore not surprising
that practitioners and academics in these fields are reluctant to adopt
AI technologies they cannot trust. Deep neural networks, for example,
are generally considered as black boxes and therefore not trustworthy in
a context that demands explanations. This PhD project is focused on
exploring and developing methodologies that improve the trustworthiness
of AI and thereby enable its application in Finance and Economics.

The remainder of this extended abstract is structured as follows:
Section~\ref{sec-main} presents one of the research questions I have
investigated during the first months of my PhD: how do counterfactual
explanations handle dynamics? Section~\ref{sec-related} places this work
in the broader context of my research.

\hypertarget{sec-main}{%
\section{Dynamics in Algorithmic Recourse}\label{sec-main}}

\textbf{Counterfactual explanations} (CE) explain how inputs into a
model need to change for it to produce different outputs. They are
intuitive, simple and intrinsically linked to the potential outcome
framework for causal inference, which social scientists are familiar
with. Counterfactual explanations that involve realistic and actionable
changes can be used for the purpose of \textbf{Algorithmic Recourse}
(AR) to help individuals who face adverse outcomes. An example relevant
to the Finance and Economics domain is consumer credit: in this context
AR can be used to guide individuals in improving their creditworthiness,
should they have previously been denied access to credit based on an
automated decision-making system.

Existing work on CE and AR has largely been limited to the static
setting: given some classifier \(M: \mathcal{X} \mapsto \mathcal{Y}\) we
are interested in finding close (Wachter, Mittelstadt, and Russell
2017), actionable (Ustun, Spangher, and Liu 2019), realistic Schut et
al. (2021), sparse, diverse (Mothilal, Sharma, and Tan 2020) and ideally
causally founded counterfactual explanations (Karimi, Schölkopf, and
Valera 2021) for some individual \(x\). The ability of counterfactual
explanations to handle dynamics like data and model shifts remains a
largely unexplored research challenge at this point (Verma, Dickerson,
and Hines 2020). Only one recent work considers the implications of
\textbf{exogenous} domain and model shifts (Upadhyay, Joshi, and
Lakkaraju 2021). The authors propose a simple minimax objective, that
minimizes the counterfactual loss function for a maximal model shift.
They show that their approach yields more robust counterfactuals in this
context than existing approaches.

This project investigates \textbf{endogenous} domain and model shifts,
that is shifts that occur when AR is actually implemented by a
proportion of individuals and the classifier is updated in response.
Figure~\ref{fig-dynamics} illustrates this idea for a binary problem
involving a probabilistic classifier and the counterfactual generator
proposed by Wachter, Mittelstadt, and Russell (2017): the implementation
of AR for a subset of individuals leads to a domain shift (b), which in
turn triggers a model shift (c). As this game of implementing AR and
updating the classifier is repeated, the decision boundary moves away
from training samples that were originally in the target class (d).

These dynamics may be problematic. As the decision boundary moves in the
direction of the non-target class, counterfactual paths become shorter:
in the loan example, individuals that previously would have been denied
credit based on their input features are suddenly considered as
creditworthy. Average default risk across all borrowers can therefore be
expected to increase. Conversely, lenders that anticipate such dynamics
may choose to deny credit to individuals that have implemented AR,
thereby compromising the validity of AR.

To the best of my knowledge this is the first work investigating
endogenous dynamics in AR. Through future experiments I want to
investigate how this phenomenon plays out across different benchmark
datasets including German credit, Boston Housing and COMPAS.\footnote{These
  benchmark datasets have their issues and controversies, which is one
  of the challenges I would like to discuss at AIES.} Furthermore, I
want to assess to what extent the magnitude and direction of domain and
model shifts depends on the choice of the counterfactual generator. To
this end, I am currently supervising a group of undergraduate students,
who are tackling some of these tasks in their final-year research
project.

\begin{figure}

{\centering \includegraphics[width=0.45\textwidth,height=\textheight]{www/poc.png}

}

\caption{\label{fig-dynamics}Dynamics in Algorithmic Recourse: we have a
simple Bayesian model trained for binary classification (a); the
implementation of AR for a random subset of individuals leads to a
domain shift (b); as the classifier is retrained we observe a model
shift (c); as this process is repeated, the decision boundary moves away
from the target class (d).}

\end{figure}

\hypertarget{sec-related}{%
\section{Related and Future Work}\label{sec-related}}

\hypertarget{benchmarking-ce-in-julia}{%
\subsection{Benchmarking CE in Julia}\label{benchmarking-ce-in-julia}}

Until recently there existed only one open-source library that provides
a unifying approach to generate and benchmark counterfactual
explanations for Python models (Pawelczyk et al. 2021). To address this
limitation I have developed
\href{https://www.paltmeyer.com/CounterfactualExplanations.jl/stable/}{CounterfactualExplanations.jl}:
a Julia package that can be used to generate counterfactual explanations
for models developed and trained not only in Julia, but also in other
popular programming languages. The package and companion paper are
pending acceptance for a main talk at
\href{https://juliacon.org/2022/}{JuliaCon '22}.

\hypertarget{probabilistic-methods-for-realistic-ce}{%
\subsection{Probabilistic Methods for Realistic
CE}\label{probabilistic-methods-for-realistic-ce}}

To ensure that the generated counterfactuals are realistic it helps to
understand which input-output pairs are likely to occur under the data
generating process. To this end, previous work has either relied on
generative models or restricted the analysis to probabilistic
classifiers that incorporate uncertainty in their predictions. While the
former approach is more generally applicable, the latter is
computationally more efficient. In future work, I want to explore how
recent advances in post-hoc uncertainty quantification, most notably
Laplace Redux (Daxberger et al. 2021), can be leveraged to generate
realistic and unambiguous counterfactual explanations for any
model.\footnote{For some initial work on this see my Julia
  implementation of Laplace Redux:
  \href{https://www.paltmeyer.com/BayesLaplace.jl/dev/}{BayesLaplace.jl}.}
With respect to the work-in-progress presented here, I expect that these
efforts may help in mitigating endogenous domain and model shifts.

\hypertarget{references}{%
\section*{References}\label{references}}
\addcontentsline{toc}{section}{References}

\hypertarget{refs}{}
\begin{CSLReferences}{1}{0}
\leavevmode\vadjust pre{\hypertarget{ref-daxberger2021laplace}{}}%
Daxberger, Erik, Agustinus Kristiadi, Alexander Immer, Runa Eschenhagen,
Matthias Bauer, and Philipp Hennig. 2021. {``Laplace Redux-Effortless
Bayesian Deep Learning.''} \emph{Advances in Neural Information
Processing Systems} 34.

\leavevmode\vadjust pre{\hypertarget{ref-joshi2019towards}{}}%
Joshi, Shalmali, Oluwasanmi Koyejo, Warut Vijitbenjaronk, Been Kim, and
Joydeep Ghosh. 2019. {``Towards Realistic Individual Recourse and
Actionable Explanations in Black-Box Decision Making Systems.''}
\emph{arXiv Preprint arXiv:1907.09615}.

\leavevmode\vadjust pre{\hypertarget{ref-karimi2021algorithmic}{}}%
Karimi, Amir-Hossein, Bernhard Schölkopf, and Isabel Valera. 2021.
{``Algorithmic Recourse: From Counterfactual Explanations to
Interventions.''} In \emph{Proceedings of the 2021 ACM Conference on
Fairness, Accountability, and Transparency}, 353--62.

\leavevmode\vadjust pre{\hypertarget{ref-mothilal2020explaining}{}}%
Mothilal, Ramaravind K, Amit Sharma, and Chenhao Tan. 2020.
{``Explaining Machine Learning Classifiers Through Diverse
Counterfactual Explanations.''} In \emph{Proceedings of the 2020
Conference on Fairness, Accountability, and Transparency}, 607--17.

\leavevmode\vadjust pre{\hypertarget{ref-pawelczyk2021carla}{}}%
Pawelczyk, Martin, Sascha Bielawski, Johannes van den Heuvel, Tobias
Richter, and Gjergji Kasneci. 2021. {``Carla: A Python Library to
Benchmark Algorithmic Recourse and Counterfactual Explanation
Algorithms.''} \emph{arXiv Preprint arXiv:2108.00783}.

\leavevmode\vadjust pre{\hypertarget{ref-schut2021generating}{}}%
Schut, Lisa, Oscar Key, Rory Mc Grath, Luca Costabello, Bogdan
Sacaleanu, Yarin Gal, et al. 2021. {``Generating Interpretable
Counterfactual Explanations by Implicit Minimisation of Epistemic and
Aleatoric Uncertainties.''} In \emph{International Conference on
Artificial Intelligence and Statistics}, 1756--64. PMLR.

\leavevmode\vadjust pre{\hypertarget{ref-upadhyay2021towards}{}}%
Upadhyay, Sohini, Shalmali Joshi, and Himabindu Lakkaraju. 2021.
{``Towards Robust and Reliable Algorithmic Recourse.''} \emph{arXiv
Preprint arXiv:2102.13620}.

\leavevmode\vadjust pre{\hypertarget{ref-ustun2019actionable}{}}%
Ustun, Berk, Alexander Spangher, and Yang Liu. 2019. {``Actionable
Recourse in Linear Classification.''} In \emph{Proceedings of the
Conference on Fairness, Accountability, and Transparency}, 10--19.

\leavevmode\vadjust pre{\hypertarget{ref-verma2020counterfactual}{}}%
Verma, Sahil, John Dickerson, and Keegan Hines. 2020. {``Counterfactual
Explanations for Machine Learning: A Review.''} \emph{arXiv Preprint
arXiv:2010.10596}.

\leavevmode\vadjust pre{\hypertarget{ref-wachter2017counterfactual}{}}%
Wachter, Sandra, Brent Mittelstadt, and Chris Russell. 2017.
{``Counterfactual Explanations Without Opening the Black Box: Automated
Decisions and the GDPR.''} \emph{Harv. JL \& Tech.} 31: 841.

\end{CSLReferences}



\end{document}
